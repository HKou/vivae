\section{Measurements}

This section presents methods used for measurement of the simulation and obtaing the results. Most common way of doing it is a fitness function that returns a given quality of the simulation. the fitness function is used afterwards. The fitness classes are placed in \package{vivae.fitness} package.


\subsection{Fitness Functions}

The concept of fitness functions contains a set of classes, where each class has \method{getFitness()} method that returns the actual value of the fitness measure specified in the class.

The basic usage calls the particular fitness class conructor before the experiment starts to initialize it. The \method{getFitness()} class is called after the experiment ends to get the fitness value.

The fitness functions not neccessarily have to be called after the experiment ends.

\subsubsection{Average Speed}

Average speed fitness \method{getFitness()} method returns the average speed of all agents in the simulation. Each agent represented by an offspring of \class{Robot} class has an odometer $o$, which accumulates the distance it travels each simulation step. The overall fitness is divided by a number of agents $A$ in the simulation and number of simulation steps:

\begin{equation}
f_v = \frac{\displaystyle\sum_{\forall a}^{A} o_a}{A \cdot steps}
\end{equation}

\subsubsection{Movables on Top}

\class{MovablesOnTop} fitness class saves sum of distances $d$ of all \class{Movable} offspring classes $M$ in simulation first (in the constructor). At the end of simulation ($t_{end}$) a sum vertical positions is subtracted from the saved value divided by a number of the movable obstacles and normalized to the arena height according to the following equation:

\begin{equation}
d = \displaystyle\sum_{m}^{M} y_m,\qquad
f_m = \frac{d^{t_0} - d^{t_{end}}}{M \cdot Y_{m}}
\end{equation}

where $y_m$ is the vertical position of the movable object center and $Y_m$ is the arena height.

